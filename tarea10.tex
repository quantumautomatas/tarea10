\documentclass{article}

% formato
\usepackage[margin = 1.5cm, letterpaper]{geometry}
\usepackage[spanish,es-noshorthands]{babel}
\usepackage[utf8]{inputenc}
%tablas
\usepackage{graphicx}

%formato ecuaciones
\usepackage{amsmath}

% símbolos
\usepackage{amssymb}

% manejo de tablas
\usepackage{float}

% autómatas
\usepackage{tikz}
\usetikzlibrary{automata, positioning, arrows}

\begin{document}
    \title{
        Autómatas y Lenguajes formales \\
        Ejercicio Semanal 10
    }

    \author{
        Sandra del Mar Soto Corderi \\
        Edgar Quiroz Castañeda
    }

    \date{
        25 de abril del 2019
    }
    
    \maketitle

    \begin{enumerate}
        \item {
        Dado el lenguaje L definido como sigue:\\
        \begin{equation*}
        	L = \{a^nb^mc^k | m \neq n \ o \   m \neq k \}
        \end{equation*}
        
        
        \begin{enumerate}
        	\item {
        	Diseña un Autómata de pila que acepte L.\\
         	Primero hay que diseñar un autómata que acepte a $\{a^nb^mc^k\}$.
         	Este sería 
         	\[M_1 = \langle Q_1 , \Sigma, \Gamma_1 , \delta_1a, q_1 , Z_0,  F_1\rangle\]
         
	         Donde:
	         
	         \begin{itemize}
	         	\item {
	         		$Q_1 = \{q_1, q_2, q_3, q_4, q_5\}$
	         	}
	         	\item {
	         		$\Sigma = \{a, b,c\}$
	         	}
	         	\item {
	         		$\Gamma_1 = \{Z_0, A\}$
	         	}
	         	        					 
				\item {     		
					\begin{align*}
						&\delta (q_1, a, Z_0) = (q_1, AZ_0)
						&\delta (q_1, a, A) = (q_1, AA) \\
						&\delta (q_1, b, Z_0) = (q_4, \epsilon)
						&\delta (q_1, b, A) = (q_2, \epsilon) \\
						&\delta (q_1, c, A) = (q_5, \epsilon)
						&\delta (q_1, \epsilon, A) = (q_3, \epsilon) \\
						&\delta (q_2, b, A) = (q_2, \epsilon)
						&\delta (q_2, b, Z_0) = (q_4, \epsilon) \\
						&\delta (q_2, c, A) = (q_5, \epsilon)
						&\delta (q_2, \epsilon, A) = (q_3, \epsilon) \\
						&\delta (q_4, b, Z_0) = (q_4, Z_0)
						&\delta (q_4, b, A) = (q_4, \epsilon) \\
						&\delta (q_4, c, Z_0) = (q_5, Z_0)
						&\delta (q_4, c, A) = (q_5, \epsilon) \\
						&\delta (q_5, c, Z_0) = (q_5, Z_0)
						&\delta (q_5, c, A) = (q_5, \epsilon)
	         		\end{align*}
				}
				 
	         	
	         	\item {
	         		$q_1$ es el estado inicial.
	         	}
	         	\item {
	         		$Z_0$ es el símbolo al fondo de la pila.
	         	}
	         	\item {
	         		$F_1 = \{q_3, q_4, q_5\}$
	         	}
	         \end{itemize}
	     
	     	Luego, hay que hacer otro autómata 
	     	\[M_2 = \langle Q_2 , \Sigma, \Gamma_2 , \delta_2, q_6 , Z_0,
	     	F_2\rangle\]
	     	Para reconocer $\{a^nb^mc^k | m \neq k\}$
	     	Dado por 
	     	\begin{itemize}
	     		\item {
	     			$Q_2 = \{q_6, q_7, q_8, q_9\}$
     			}
     		
     			\item {
     				$\Sigma = \{a, b, c\}$
     			}
     			\item {
     				$\Gamma_2 = \{B, Z_0\}$
     			}
	     		\item {
	     			\begin{align*}
	     			&\delta (q_6, a, Z_0) = (q_6, Z_0)
	     			&\delta (q_6, b, Z_0) = (q_6, BZ_0) \\
	     			&\delta (q_6, b, B) = (q_6, BB)
	     			&\delta (q_6, c, Z_0) = (q_8, \epsilon) \\
	     			&\delta (q_6, c, B) = (q_7, \epsilon)
	     			&\delta (q_6, \epsilon, B) = (q_8, \epsilon) \\
	     			&\delta (q_7, c, B) = (q_7, \epsilon)
	     			&\delta (q_7, c, Z_0) = (q_9, \epsilon) \\
	     			&\delta (q_7, \epsilon, B) = (q_2, \epsilon)
	     			&\delta (q_9, c, Z_0) = (q_9, Z_0) \\
	     			&\delta (q_9, c, B) = (q_9, \epsilon)
	     			\end{align*}
	     		}
	     		\item {
	     			$q_6$ es el estado inicial.
	     		}
	     		\item {
	     			$Z_0$ es el símbolo al fondo de la pila.
	     		}
	     		\item {
	     			$F_2 = \{q_8, q_9\}$
	     		}
	     	\end{itemize}
	            
	        Entonces el autómata 
	        \[M = \langle Q , \Sigma, \Gamma , \delta, q_0 , Z_0,
	        F\rangle\]
	        Que acepta al lenguaje deseado está dado por
	        
	        \begin{itemize}
	        	\item {
	        		$Q = \{q_0\} \cup Q_1 \cup Q_2$
	        	}
	        	
	        	\item {
	        		$\Sigma = \{a, b, c\}$
	        	}
	        	\item {
	        		$\Gamma = \{A, B, Z_0\}$
	        	}
	        	\item {
	        		\begin{align*}
						\delta (q_0, \epsilon, Z_0) &= \{(q_0, Z_0), (q_6, Z_0)\} \\
						\delta (p, s, \gamma) &= 
						\begin{cases}
							&\delta_1(p, s, \gamma) \text{ si } p \in Q_1 \\
							&\delta_2(p, s, \gamma) \text{ si } p \in Q_2 \\
						\end{cases}
					\end{align*}
	        	}
	        	\item {
	        		$q_0$ es el estado inicial.
	        	}
	        	\item {
	        		$Z_0$ es el símbolo al fondo de la pila.
	        	}
	        	\item {
	        		$F = F_1 \cup F_2$
	        	}
	        \end{itemize}
	        }
        	\item{
            Construye un autómata de pila que acepte el lenguaje, a partir del
            AP del inciso anterior. Usando los algoritmos vistos en clase para
            cambiar el criterio de aceptación del autómata.\\
            El autómata resultante sería 
            \[M' = \langle Q \cup \{p_0, p\} , \Sigma, \Gamma \cup \{N_0\} , \delta', P_0 , N_0, \varnothing \rangle\]
            Con $\delta'$ igual a $\delta$, excepto por estas transiciones nuevas
            \begin{align*}
            	&\delta(p_0, \epsilon, N_0) = \{(q_0, Z_0N_0)\}\\
				&\delta(q_3, \epsilon, s) = \{(p, s)\} \\
				&\delta(q_4, \epsilon, s) = \{(p, s)\} \\
				&\delta(q_5, \epsilon, s) = \{(p, s)\} \\
				&\delta(q_8, \epsilon, s) = \{(p, s)\} \\
				&\delta(q_9, \epsilon, s) = \{(p, s)\} \\
            	&\delta(p, \epsilon, s) = \{(p,\epsilon)\}
			\end{align*}
			Donde $s\in \Gamma \cup \{N_0\}$ es arbitrario.
           }   	
        	\item{
        	Muestra la ejecución formal, en ambos autómatas, de las cadenas:
        	
        	\begin{itemize}
        		\item {
        		aabbccc\\
        		
        		
        		}
        		\item {
        		aabbcc\\
        		
        		}
        	\end{itemize}
        		
        	}
        
        \end{enumerate}
    	}
    \end{enumerate}
\end{document}