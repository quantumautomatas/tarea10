\documentclass{article}

% formato
\usepackage[margin = 1.5cm, letterpaper]{geometry}
\usepackage[spanish,es-noshorthands]{babel}
\usepackage[utf8]{inputenc}
%tablas
\usepackage{graphicx}

%formato ecuaciones
\usepackage{amsmath}

% símbolos
\usepackage{amssymb}

% manejo de tablas
\usepackage{float}

% autómatas
\usepackage{tikz}
\usetikzlibrary{automata, positioning, arrows}

\begin{document}
    \title{
        Autómatas y Lenguajes formales \\
        Ejercicio Semanal 10
    }

    \author{
        Sandra del Mar Soto Corderi \\
        Edgar Quiroz Castañeda
    }

    \date{
        26 de abril del 2019
    }
    
    \maketitle

    \begin{enumerate}
        \item {
        Dado el lenguaje L definido como sigue:\\
        \begin{equation*}
        	L = \{a^nb^mc^k | m \neq n \ o \   m \neq k \}
        \end{equation*}
        
        
        \begin{enumerate}
        	\item {
        	Diseña un Autómata de pila que acepte L.\\
         	Primero hay que diseñar un autómata que acepte a $\{a^nb^mc^k\}$.
         	Este sería 
         	\[M_1 = \langle Q_1 , \Sigma, \Gamma_1 , \delta_1a, q_1 , Z_0,  F_1\rangle\]
         
	         Donde:
	         
	         \begin{itemize}
	         	\item {
	         		$Q_1 = \{q_1, q_2, q_3, q_4, q_5\}$
	         	}
	         	\item {
	         		$\Sigma = \{a, b,c\}$
	         	}
	         	\item {
	         		$\Gamma_1 = \{Z_0, A\}$
	         	}
	         	        					 
				\item {     		
					\begin{align*}
						&\delta (q_1, a, Z_0) = (q_1, AZ_0)
						&\delta (q_1, a, A) = (q_1, AA) \\
						&\delta (q_1, b, Z_0) = (q_4, \epsilon)
						&\delta (q_1, b, A) = (q_2, \epsilon) \\
						&\delta (q_1, c, A) = (q_5, \epsilon)
						&\delta (q_1, \epsilon, A) = (q_3, \epsilon) \\
						&\delta (q_2, b, A) = (q_2, \epsilon)
						&\delta (q_2, b, Z_0) = (q_4, \epsilon) \\
						&\delta (q_2, c, A) = (q_5, \epsilon)
						&\delta (q_2, \epsilon, A) = (q_3, \epsilon) \\
						&\delta (q_4, b, Z_0) = (q_4, Z_0)
						&\delta (q_4, b, A) = (q_4, \epsilon) \\
						&\delta (q_4, c, Z_0) = (q_5, Z_0)
						&\delta (q_4, c, A) = (q_5, \epsilon) \\
						&\delta (q_5, c, Z_0) = (q_5, Z_0)
						&\delta (q_5, c, A) = (q_5, \epsilon)
	         		\end{align*}
				}
				 
	         	
	         	\item {
	         		$q_1$ es el estado inicial.
	         	}
	         	\item {
	         		$Z_0$ es el símbolo al fondo de la pila.
	         	}
	         	\item {
	         		$F_1 = \{q_3, q_4, q_5\}$
	         	}
	         \end{itemize}
	     
	     	Luego, hay que hacer otro autómata 
	     	\[M_2 = \langle Q_2 , \Sigma, \Gamma_2 , \delta_2, q_6 , Z_0,
	     	F_2\rangle\]
	     	Para reconocer $\{a^nb^mc^k | m \neq k\}$
	     	Dado por 
	     	\begin{itemize}
	     		\item {
	     			$Q_2 = \{q_6, q_7, q_8, q_9\}$
     			}
     		
     			\item {
     				$\Sigma = \{a, b, c\}$
     			}
     			\item {
     				$\Gamma_2 = \{B, Z_0\}$
     			}
	     		\item {
	     			\begin{align*}
	     			&\delta (q_6, a, Z_0) = (q_6, Z_0)
	     			&\delta (q_6, b, Z_0) = (q_6, BZ_0) \\
	     			&\delta (q_6, b, B) = (q_6, BB)
	     			&\delta (q_6, c, Z_0) = (q_8, \epsilon) \\
	     			&\delta (q_6, c, B) = (q_7, \epsilon)
	     			&\delta (q_6, \epsilon, B) = (q_8, \epsilon) \\
	     			&\delta (q_7, c, B) = (q_7, \epsilon)
	     			&\delta (q_7, c, Z_0) = (q_9, \epsilon) \\
	     			&\delta (q_7, \epsilon, B) = (q_2, \epsilon)
	     			&\delta (q_9, c, Z_0) = (q_9, Z_0)
	     			&\delta (q_9, c, B) = (q_9, \epsilon)
	     			\end{align*}
	     		}
	     		\item {
	     			$q_6$ es el estado inicial.
	     		}
	     		\item {
	     			$Z_0$ es el símbolo al fondo de la pila.
	     		}
	     		\item {
	     			$F_2 = \{q_8, q_9\}$
	     		}
	     	\end{itemize}
	            
	        Entonces el autómata 
	        \[M = \langle Q , \Sigma, \Gamma , \delta, q_0 , Z_0,
	        F\rangle\]
	        Que acepta al lenguaje deseado está dado por
	        
	        \begin{itemize}
	        	\item {
	        		$Q = \{q_0\} \cup Q_1 \cup Q_2$
	        	}
	        	
	        	\item {
	        		$\Sigma = \{a, b, c\}$
	        	}
	        	\item {
	        		$\Gamma = \{A, B, Z_0\}$
	        	}
	        	\item {
	        		\begin{align*}
	        		&\delta (q_0, \epsilon, Z_0) = \{(q_1, Z_0),
	        		(q_6, Z_0)\}\\
	        		&\delta (p, s, \gamma) = \delta_1 (p, s, \gamma) \cup
	        		\delta_2 (p, s, \gamma) 		
	        		\end{align*}
	        		Donde $p$ es cualquier estado que no sea $q_0$, $s$ un símbolo de $\Sigma$	y $\gamma$ cualquier símbolo de $\Gamma$.
	        	}
	        	\item {
	        		$q_0$ es el estado inicial.
	        	}
	        	\item {
	        		$Z_0$ es el símbolo al fondo de la pila.
	        	}
	        	\item {
	        		$F = F_1 \cup F_2$
	        	}
	        \end{itemize}
	        }
        	\item{
            Construye un autómata de pila que acepte el lenguaje, a partir del
            AP del inciso anterior. Usando los algoritmos vistos en clase para
            cambiar el criterio de aceptación del autómata.\\
            El autómata resultante sería 
            \[M' = \langle Q \cup \{p_0, p\} , \Sigma, \Gamma \cup \{N_0\} , \delta', P_0 , N_0, \varnothing \rangle\]
            Con $\delta'$ dada por
            \begin{align*}
            	&\delta(p_0, \epsilon, N_0) = \{q_0, Z_0N_0\}\\
            	&(p,s) \in \delta(q_f, \epsilon, s), \text{ para todo } q_f \in F, s \in \Gamma \cup \{N_0\}\\
            	&\delta(p, \epsilon, s) = \{(p,\epsilon)\}
            \end{align*}
           }   	
        	\item{
        	Muestra la ejecución formal, en ambos autómatas, de las cadenas:
        	
        	Ejecución formal en el autómata que acepta por estado final:\\
        	\begin{itemize}
        		\item {
        		aabbccc\\
        		$\langle q_0, aabbccc, Z_0 \rangle \vdash$ como el autómata es no determinista, hay dos opciones para esta transición, desarrollaremos ambas:\\
        		
        		\begin{itemize}
        			\item {
        			$\langle q_1, aabbccc, Z_0 \rangle \vdash \langle q_1, abbccc, AZ_0 \rangle \vdash \langle q_1, bbccc, AAZ_0 \rangle \vdash  \langle q_2, bccc, AZ_0 \rangle$ 
        			}
        			\item {
        			$\vdash \langle q_2, ccc, \epsilon Z_0 \rangle$  NO ACEPTA\\
        			}
        		\end{itemize}
        	
        		\begin{itemize}
        			\item {
        				$\langle q_6, aabbccc, Z_0 \rangle \vdash \langle q_6, abbccc, Z_0 \rangle \vdash \langle q_6, bbccc, Z_0 \rangle \vdash  \langle q_6, bccc, BZ_0 \rangle \vdash  \langle q_6, ccc, BBZ_0 \rangle \vdash \langle q_7, cc, BZ_0 \rangle \vdash \langle q_7, c, Z_0 \rangle$ 
        			}
        			\item {
        			$\vdash \langle q_9, \epsilon, \epsilon Z_0 \rangle$ ACEPTA porque $q_9 \in F$\\
        			}
        		
        			Como un cómputo acepto la cadena, entonces el autómata la acepta.\\
        		\end{itemize}
        		
        		}
        		\item {
        		aabbcc\\
        		
        		$\langle q_0, aabbcc, Z_0 \rangle \vdash$ como el autómata es no determinista, hay dos opciones para esta transición, desarrollaremos ambas:\\
        		
        		\begin{itemize}
        			\item {
        				$\langle q_1, aabbcc, Z_0 \rangle \vdash \langle q_1, abbcc, AZ_0 \rangle \vdash \langle q_1, bbcc, AAZ_0 \rangle \vdash \langle q_2, bcc, AZ_0 \rangle $ 
        			}
        			\item {
        				$\vdash \langle q_2, cc, Z_0 \rangle$  NO ACEPTA\\
        			}
        		\end{itemize}
        		
        		\begin{itemize}
        		\item {
        			$\langle q_6, aabbcc, Z_0 \rangle \vdash \langle q_6, abbcc, Z_0 \rangle \vdash \langle q_6, bbcc, Z_0 \rangle \vdash  \langle q_6, bccc, BZ_0 \rangle \vdash  \langle q_6, cc, BBZ_0 \rangle \vdash \langle q_7, c, BZ_0 \rangle$ 
        		}
        		\item {
        			$\vdash \langle q_7, \epsilon, Z_0 \rangle$ NO ACEPTA\\
        		}
        		\end{itemize}
        		Como ningún cómputo acepto la cadena, entonces el autómata la rechaza.\\
        		
        		}
        	\end{itemize}
        		
        	}
        
        \end{enumerate}
    	}
    \end{enumerate}
\end{document}