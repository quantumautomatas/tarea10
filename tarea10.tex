\documentclass{article}

% formato
\usepackage[margin = 1.5cm, letterpaper]{geometry}
\usepackage[spanish,es-noshorthands]{babel}
\usepackage[utf8]{inputenc}
%tablas
\usepackage{graphicx}

%formato ecuaciones
\usepackage{amsmath}

% símbolos
\usepackage{amssymb}

% manejo de tablas
\usepackage{float}

% autómatas
\usepackage{tikz}
\usetikzlibrary{automata, positioning, arrows}

\begin{document}
    \title{
        Autómatas y Lenguajes formales \\
        Ejercicio Semanal 10
    }

    \author{
        Sandra del Mar Soto Corderi \\
        Edgar Quiroz Castañeda
    }

    \date{
        25 de abril del 2019
    }
    
    \maketitle

    \begin{enumerate}
        \item {
        Dado el lenguaje L definido como sigue:\\
        \begin{equation*}
        	L = \{a^nb^mc^k | m \neq n \ o \   m \neq k \}
        \end{equation*}
        
        
        \begin{enumerate}
        	\item {
        	Diseña un Autómata de pila que acepte L.\\
         El autómata es \[M = \langle Q , \Sigma, \Gamma , \delta, q_0 , Z_0,  F\rangle\]
         
         Donde:
         
         \begin{itemize}
         	\item {
         		$Q = \{q_0, q_1, q_2, q_3, q_4\}$
         	}
         	\item {
         		$\Sigma = \{a, b,c\}$
         	}
         	\item {
         		$\Gamma = \{Z_0, A, B, C\}$\\
         	}
         	
         	
         	\item {
         	        		
        		\begin{align*}
         			\delta (q_0, a , Z_0) &= \{(q_0, AZ_0)\}\\
         			\delta (q_0, a , A) &= \{(q_0, AA)\}\\
         			\delta (q_0, b , A) &= \{(q_1, \epsilon), (q_2, BA)\}\\
         			\delta (q_1, b , A) &= \{(q_1, \epsilon)\}\\
         			\delta (q_1, c , A) &= \{(q_4, \epsilon)\}\\
         			\delta (q_1, b , Z_0) &= \{(q_4, \epsilon)\}\\
         			\delta (q_2, b , B) &= \{(q_2, BB)\}\\
         			\delta (q_2, c , B) &= \{(q_3, \epsilon)\}\\
         			\delta (q_3, c , B) &= \{(q_3, \epsilon)\}\\
         			\delta (q_3, c , A) &= \{(q_4, \epsilon)\}\\
         			\delta (q_3, \epsilon , B) &= \{(q_4, \epsilon)\}\\
         		\end{align*}
         	}
         	
         	\item {
         		$q_0$ es el estado inicial.
         	}
         	\item {
         		$Z_0$ es el símbolo al fondo de la pila.
         	}
         	\item {
         		$F = q_4$
         	}
         \end{itemize}
            
            
        	}
        	\item{
            Construye un autómata de pila que acepte el lenguaje, a partir del
            AP del inciso anterior. Usando los algoritmos vistos en clase para
            cambiar el criterio de aceptación del autómata.\\
            
        	}
        	\item{
        	Muestra la ejecución formal, en ambos autómatas, de las cadenas:
        	
        	\begin{itemize}
        		\item {
        		aabbccc\\
        		
        		
        		}
        		\item {
        		aabbcc\\
        		
        		}
        	\end{itemize}
        		
        	}
        
        \end{enumerate}
    	}
    \end{enumerate}
\end{document}