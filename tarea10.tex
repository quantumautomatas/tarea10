\documentclass{article}

% formato
\usepackage[margin = 1.5cm, letterpaper]{geometry}
\usepackage[spanish,es-noshorthands]{babel}
\usepackage[utf8]{inputenc}
%tablas
\usepackage{graphicx}

%formato ecuaciones
\usepackage{amsmath}

% símbolos
\usepackage{amssymb}

% manejo de tablas
\usepackage{float}

% autómatas
\usepackage{tikz}
\usetikzlibrary{automata, positioning, arrows}

\begin{document}
    \title{
        Autómatas y Lenguajes formales \\
        Ejercicio Semanal 10
    }

    \author{
        Sandra del Mar Soto Corderi \\
        Edgar Quiroz Castañeda
    }

    \date{
        25 de abril del 2019
    }
    
    \maketitle

    \begin{enumerate}
        \item {
        Dado el lenguaje L definido como sigue:\\
        \begin{equation*}
        	L = \{a^nb^mc^k | m \neq n \ o \   m \neq k \}
        \end{equation*}
        
        
        \begin{enumerate}
        	\item {
        	Diseña un Autómata de pila que acepte L.\\
        	
            
            
        	}
        	\item{
            Construye un autómata de pila que acepte el lenguaje, a partir del
            AP del inciso anterior. Usando los algoritmos vistos en clase para
            cambiar el criterio de aceptación del autómata.\\
            
        	}
        	\item{
        	Muestra la ejecución formal, en ambos autómatas, de las cadenas:
        	
        	\begin{itemize}
        		\item {
        		aabbccc\\
        		
        		
        		}
        		\item {
        		aabbcc\\
        		
        		}
        	\end{itemize}
        		
        	}
        
        \end{enumerate}
    	}
    \end{enumerate}
\end{document}